\documentclass[]{elsarticle} %review=doublespace preprint=single 5p=2 column
%%% Begin My package additions %%%%%%%%%%%%%%%%%%%
\usepackage[hyphens]{url}

  \journal{Journal of Archaeological Science} % Sets Journal name


\usepackage{lineno} % add
\providecommand{\tightlist}{%
  \setlength{\itemsep}{0pt}\setlength{\parskip}{0pt}}

\bibliographystyle{elsarticle-harv}
\biboptions{sort&compress} % For natbib
\usepackage{graphicx}
\usepackage{booktabs} % book-quality tables
%%%%%%%%%%%%%%%% end my additions to header

\usepackage[T1]{fontenc}
\usepackage{lmodern}
\usepackage{amssymb,amsmath}
\usepackage{ifxetex,ifluatex}
\usepackage{fixltx2e} % provides \textsubscript
% use upquote if available, for straight quotes in verbatim environments
\IfFileExists{upquote.sty}{\usepackage{upquote}}{}
\ifnum 0\ifxetex 1\fi\ifluatex 1\fi=0 % if pdftex
  \usepackage[utf8]{inputenc}
\else % if luatex or xelatex
  \usepackage{fontspec}
  \ifxetex
    \usepackage{xltxtra,xunicode}
  \fi
  \defaultfontfeatures{Mapping=tex-text,Scale=MatchLowercase}
  \newcommand{\euro}{€}
\fi
% use microtype if available
\IfFileExists{microtype.sty}{\usepackage{microtype}}{}
\usepackage{color}
\usepackage{fancyvrb}
\newcommand{\VerbBar}{|}
\newcommand{\VERB}{\Verb[commandchars=\\\{\}]}
\DefineVerbatimEnvironment{Highlighting}{Verbatim}{commandchars=\\\{\}}
% Add ',fontsize=\small' for more characters per line
\usepackage{framed}
\definecolor{shadecolor}{RGB}{248,248,248}
\newenvironment{Shaded}{\begin{snugshade}}{\end{snugshade}}
\newcommand{\KeywordTok}[1]{\textcolor[rgb]{0.13,0.29,0.53}{\textbf{{#1}}}}
\newcommand{\DataTypeTok}[1]{\textcolor[rgb]{0.13,0.29,0.53}{{#1}}}
\newcommand{\DecValTok}[1]{\textcolor[rgb]{0.00,0.00,0.81}{{#1}}}
\newcommand{\BaseNTok}[1]{\textcolor[rgb]{0.00,0.00,0.81}{{#1}}}
\newcommand{\FloatTok}[1]{\textcolor[rgb]{0.00,0.00,0.81}{{#1}}}
\newcommand{\ConstantTok}[1]{\textcolor[rgb]{0.00,0.00,0.00}{{#1}}}
\newcommand{\CharTok}[1]{\textcolor[rgb]{0.31,0.60,0.02}{{#1}}}
\newcommand{\SpecialCharTok}[1]{\textcolor[rgb]{0.00,0.00,0.00}{{#1}}}
\newcommand{\StringTok}[1]{\textcolor[rgb]{0.31,0.60,0.02}{{#1}}}
\newcommand{\VerbatimStringTok}[1]{\textcolor[rgb]{0.31,0.60,0.02}{{#1}}}
\newcommand{\SpecialStringTok}[1]{\textcolor[rgb]{0.31,0.60,0.02}{{#1}}}
\newcommand{\ImportTok}[1]{{#1}}
\newcommand{\CommentTok}[1]{\textcolor[rgb]{0.56,0.35,0.01}{\textit{{#1}}}}
\newcommand{\DocumentationTok}[1]{\textcolor[rgb]{0.56,0.35,0.01}{\textbf{\textit{{#1}}}}}
\newcommand{\AnnotationTok}[1]{\textcolor[rgb]{0.56,0.35,0.01}{\textbf{\textit{{#1}}}}}
\newcommand{\CommentVarTok}[1]{\textcolor[rgb]{0.56,0.35,0.01}{\textbf{\textit{{#1}}}}}
\newcommand{\OtherTok}[1]{\textcolor[rgb]{0.56,0.35,0.01}{{#1}}}
\newcommand{\FunctionTok}[1]{\textcolor[rgb]{0.00,0.00,0.00}{{#1}}}
\newcommand{\VariableTok}[1]{\textcolor[rgb]{0.00,0.00,0.00}{{#1}}}
\newcommand{\ControlFlowTok}[1]{\textcolor[rgb]{0.13,0.29,0.53}{\textbf{{#1}}}}
\newcommand{\OperatorTok}[1]{\textcolor[rgb]{0.81,0.36,0.00}{\textbf{{#1}}}}
\newcommand{\BuiltInTok}[1]{{#1}}
\newcommand{\ExtensionTok}[1]{{#1}}
\newcommand{\PreprocessorTok}[1]{\textcolor[rgb]{0.56,0.35,0.01}{\textit{{#1}}}}
\newcommand{\AttributeTok}[1]{\textcolor[rgb]{0.77,0.63,0.00}{{#1}}}
\newcommand{\RegionMarkerTok}[1]{{#1}}
\newcommand{\InformationTok}[1]{\textcolor[rgb]{0.56,0.35,0.01}{\textbf{\textit{{#1}}}}}
\newcommand{\WarningTok}[1]{\textcolor[rgb]{0.56,0.35,0.01}{\textbf{\textit{{#1}}}}}
\newcommand{\AlertTok}[1]{\textcolor[rgb]{0.94,0.16,0.16}{{#1}}}
\newcommand{\ErrorTok}[1]{\textcolor[rgb]{0.64,0.00,0.00}{\textbf{{#1}}}}
\newcommand{\NormalTok}[1]{{#1}}
\usepackage{graphicx}
% We will generate all images so they have a width \maxwidth. This means
% that they will get their normal width if they fit onto the page, but
% are scaled down if they would overflow the margins.
\makeatletter
\def\maxwidth{\ifdim\Gin@nat@width>\linewidth\linewidth
\else\Gin@nat@width\fi}
\makeatother
\let\Oldincludegraphics\includegraphics
\renewcommand{\includegraphics}[1]{\Oldincludegraphics[width=\maxwidth]{#1}}
\ifxetex
  \usepackage[setpagesize=false, % page size defined by xetex
              unicode=false, % unicode breaks when used with xetex
              xetex]{hyperref}
\else
  \usepackage[unicode=true]{hyperref}
\fi
\hypersetup{breaklinks=true,
            bookmarks=true,
            pdfauthor={},
            pdftitle={The evolution of routes and paths in a dynamic environment},
            colorlinks=false,
            urlcolor=blue,
            linkcolor=magenta,
            pdfborder={0 0 0}}
\urlstyle{same}  % don't use monospace font for urls

\setcounter{secnumdepth}{0}
% Pandoc toggle for numbering sections (defaults to be off)
\setcounter{secnumdepth}{0}
% Pandoc header



\begin{document}
\begin{frontmatter}

  \title{The evolution of routes and paths in a dynamic environment}
    \author[Arizona State University]{Nicolas Gauthier\corref{c1}}
   \ead{Nicolas.Gauthier@asu.edu} 
   \cortext[c1]{Corresponding Author}
      \address[Arizona State University]{School of Human Evolution and Social Change, S. Caddy Mall, Tempe, AZ,
Zip}
  
  \begin{abstract}
  
  \end{abstract}
  
 \end{frontmatter}

\section{Introduction}\label{introduction}

Archaeology is embracing the form of constrained spatial interaction
modeling developed by Wilson. These Boltzmann-Lotka-Volterra (BLV) style
models use maximum entropy spatial interaction models to allocate flows
between a spatially structured metapopulation, and Lotka-Volterra style
consumer-resource equations to govern the growth of the populations.
These equations are able to capture the dynamic feedbacks between
settlements and the networks connecting them. Recent work in archaeology
has expanded these models to allow the networks to further evolve, as
routes that are more often used to connect important sites become
themselves important, which in turn shapes the growth of settlements
close to those routes. Past work has shown how stable routes and sets of
routes can develop in mountainous topography, where physical constraints
on movement are able to constrain the possible routes between
settlements.

Here, I extend this approach to examine how routes and paths -- the
spatial networks that connect settlements -- evolve in response to
patterns of environmental variability. Rather than leaving the carrying
capacity of our population of settlements to remain fixed, we allow it
to vary over space and in time. We examine how different patterns of
spatio-temporal change lead to different settlement patterns and spatial
network. We use a simple computatational modeling approach to facilitate
a broad range of exploration, while maintaining interpretive clarity. We
expect that different patterns, such as fixed oscillations, to lead to
different stable patterns of spatial networks, whose dynamical behavior
feeds back to influence settlement dynamics. In this way we seek to
model the potential for ``inertia'' in settlement patterns, complicating
the relationship between environmental forcing and social dynamics.
Finally we allow for bidirectional feedbacks between human populations
and the environment, exploring the potential for nonlinear
social-ecological dynamics.

We distinguish here between ``routes'' and ``paths''. The former is a
social construct, the latter are physical.

\section{Methods}\label{methods}

Parameters for the model.

\begin{Shaded}
\begin{Highlighting}[]
\NormalTok{n <-}\StringTok{ }\DecValTok{100} \CommentTok{# number of settlements}
\NormalTok{pop_start <-}\StringTok{ }\DecValTok{150} \CommentTok{# starting population per settlement}
\NormalTok{alpha <-}\StringTok{ }\FloatTok{1.05} \CommentTok{# superlinear returns to population size}
\NormalTok{beta <-}\StringTok{ }\FloatTok{0.15} \CommentTok{# distance decay parameter}
\end{Highlighting}
\end{Shaded}

We make a population of settlements. They all start with a population of
\texttt{pop\_start}, and are distributed randomly over a 100 x 100
space.

\begin{Shaded}
\begin{Highlighting}[]
\KeywordTok{set.seed}\NormalTok{(}\DecValTok{3}\NormalTok{)}
\NormalTok{net <-}\StringTok{ }\KeywordTok{tbl_graph}\NormalTok{(}\DataTypeTok{edges =} \KeywordTok{expand.grid}\NormalTok{(}\DataTypeTok{from =} \DecValTok{1}\NormalTok{:n, }\DataTypeTok{to =} \DecValTok{1}\NormalTok{:n),}
                 \DataTypeTok{nodes =} \KeywordTok{tibble}\NormalTok{(}\DataTypeTok{population =} \NormalTok{pop_start,}
                                \DataTypeTok{attractiveness =} \DecValTok{1}\NormalTok{,}
                                \DataTypeTok{x =} \KeywordTok{runif}\NormalTok{(}\DecValTok{100}\NormalTok{, }\DataTypeTok{max =} \DecValTok{100}\NormalTok{),}
                                \DataTypeTok{y =} \KeywordTok{runif}\NormalTok{(}\DecValTok{100}\NormalTok{, }\DataTypeTok{max =} \DecValTok{100}\NormalTok{))) %E>%}
\StringTok{  }\KeywordTok{filter}\NormalTok{(!}\KeywordTok{edge_is_loop}\NormalTok{()) %>%}
\StringTok{  }\KeywordTok{mutate}\NormalTok{(}\DataTypeTok{distance =} \KeywordTok{sqrt}\NormalTok{((}\KeywordTok{.N}\NormalTok{()$x[from] -}\StringTok{ }\KeywordTok{.N}\NormalTok{()$x[to]) ^}\StringTok{ }\DecValTok{2} \NormalTok{+}\StringTok{ }
\StringTok{                           }\NormalTok{(}\KeywordTok{.N}\NormalTok{()$y[from] -}\StringTok{ }\KeywordTok{.N}\NormalTok{()$y[to]) ^}\StringTok{ }\DecValTok{2}\NormalTok{)) %>%}
\StringTok{  }\KeywordTok{activate}\NormalTok{(nodes)}

\NormalTok{locations <-}\StringTok{ }\NormalTok{net %N>%}
\StringTok{  }\KeywordTok{select}\NormalTok{(x:y) %>%}
\StringTok{  }\NormalTok{as_tibble}
\end{Highlighting}
\end{Shaded}

\begin{Shaded}
\begin{Highlighting}[]
\KeywordTok{set.seed}\NormalTok{(}\DecValTok{100}\NormalTok{) }\CommentTok{# set seed for reproducability}
\NormalTok{pts <-}\StringTok{ }\KeywordTok{tibble}\NormalTok{(}\DataTypeTok{x =} \KeywordTok{sample}\NormalTok{(}\DecValTok{1}\NormalTok{:}\DecValTok{100}\NormalTok{, n, }\DataTypeTok{replace =} \OtherTok{TRUE}\NormalTok{),}
              \DataTypeTok{y =} \KeywordTok{sample}\NormalTok{(}\DecValTok{1}\NormalTok{:}\DecValTok{100}\NormalTok{, n, }\DataTypeTok{replace =} \OtherTok{TRUE}\NormalTok{))}
\end{Highlighting}
\end{Shaded}

Then figure out the euclidean distances between the settlements and pull
it all together into a \texttt{tbl\_graph} object.

\begin{Shaded}
\begin{Highlighting}[]
\NormalTok{settlements <-}\StringTok{ }\NormalTok{pts %>%}
\StringTok{  }\KeywordTok{dist}\NormalTok{(}\DataTypeTok{diag =} \OtherTok{TRUE}\NormalTok{, }\DataTypeTok{upper =} \OtherTok{TRUE}\NormalTok{) %>%}
\StringTok{  }\NormalTok{as.matrix %>%}
\StringTok{  }\NormalTok{as_tbl_graph %>%}
\StringTok{  }\KeywordTok{mutate}\NormalTok{(}\DataTypeTok{population =} \NormalTok{pop_start,}
         \DataTypeTok{attractiveness =} \NormalTok{population,}
         \DataTypeTok{x =} \NormalTok{pts$x,}
         \DataTypeTok{y =} \NormalTok{pts$y) %E>%}
\StringTok{  }\KeywordTok{rename}\NormalTok{(}\DataTypeTok{distance =} \NormalTok{weight) %>%}
\StringTok{  }\KeywordTok{filter}\NormalTok{(!}\KeywordTok{edge_is_loop}\NormalTok{())}
\end{Highlighting}
\end{Shaded}

\begin{verbatim}
## Ungrouping graph...
\end{verbatim}

\begin{verbatim}
## Using `nicely` as default layout
\end{verbatim}

\includegraphics{evolution-of-routes-and-paths_files/figure-latex/unnamed-chunk-7-1.pdf}

\subsection{Landscape Model}\label{landscape-model}

We simulate different spatial patterns of environmental variability
using the \texttt{NLMR} package.

\begin{Shaded}
\begin{Highlighting}[]
\NormalTok{null <-}\StringTok{ }\KeywordTok{matrix}\NormalTok{(}\DecValTok{0}\NormalTok{, }\DataTypeTok{nrow =} \DecValTok{100}\NormalTok{, }\DataTypeTok{ncol =} \DecValTok{100}\NormalTok{) %>%}\StringTok{ }
\StringTok{  }\KeywordTok{raster}\NormalTok{(}\DataTypeTok{xmx =} \DecValTok{100}\NormalTok{, }\DataTypeTok{ymx =} \DecValTok{100}\NormalTok{)}

\NormalTok{uniform <-}\StringTok{ }\KeywordTok{matrix}\NormalTok{(}\DecValTok{1}\NormalTok{, }\DataTypeTok{nrow =} \DecValTok{100}\NormalTok{, }\DataTypeTok{ncol =} \DecValTok{100}\NormalTok{) %>%}\StringTok{ }
\StringTok{  }\KeywordTok{raster}\NormalTok{(}\DataTypeTok{xmx =} \DecValTok{100}\NormalTok{, }\DataTypeTok{ymx =} \DecValTok{100}\NormalTok{)}

\NormalTok{random <-}\StringTok{ }\KeywordTok{nlm_random}\NormalTok{(}\DecValTok{100}\NormalTok{, }\DecValTok{100}\NormalTok{)}

\NormalTok{distance_gradient <-}\StringTok{ }\DecValTok{1}\NormalTok{-}\StringTok{ }\KeywordTok{nlm_distancegradient}\NormalTok{(}\DecValTok{100}\NormalTok{, }\DecValTok{100}\NormalTok{, }\DataTypeTok{origin =} \KeywordTok{c}\NormalTok{(}\DecValTok{0}\NormalTok{, }\DecValTok{10}\NormalTok{, }\DecValTok{0}\NormalTok{, }\DecValTok{10}\NormalTok{))}

\NormalTok{distance_gradient_center <-}\StringTok{ }\DecValTok{1}\NormalTok{-}\StringTok{ }\KeywordTok{nlm_distancegradient}\NormalTok{(}\DecValTok{100}\NormalTok{, }\DecValTok{100}\NormalTok{, }\DataTypeTok{origin =} \KeywordTok{c}\NormalTok{(}\DecValTok{49}\NormalTok{, }\DecValTok{51}\NormalTok{, }\DecValTok{49}\NormalTok{, }\DecValTok{51}\NormalTok{))}

\NormalTok{oscillation <-}\StringTok{ }\NormalTok{distance_gradient *}\StringTok{ }\NormalTok{-}\DecValTok{1} \NormalTok{+}\StringTok{ }\DecValTok{1} \NormalTok{-}\StringTok{ }\KeywordTok{nlm_distancegradient}\NormalTok{(}\DecValTok{100}\NormalTok{, }\DecValTok{100}\NormalTok{, }\DataTypeTok{origin =} \KeywordTok{c}\NormalTok{(}\DecValTok{90}\NormalTok{, }\DecValTok{100}\NormalTok{, }\DecValTok{90}\NormalTok{, }\DecValTok{100}\NormalTok{))}

\NormalTok{edge_gradient <-}\StringTok{ }\KeywordTok{nlm_edgegradient}\NormalTok{(}\DecValTok{100}\NormalTok{, }\DecValTok{100}\NormalTok{, }\DataTypeTok{direction =} \DecValTok{0}\NormalTok{)}

\NormalTok{planar_gradient <-}\StringTok{ }\KeywordTok{nlm_planargradient}\NormalTok{(}\DecValTok{100}\NormalTok{, }\DecValTok{100}\NormalTok{, }\DataTypeTok{direction =} \DecValTok{0}\NormalTok{)}

\NormalTok{grf_001 <-}\StringTok{ }\KeywordTok{nlm_gaussianfield}\NormalTok{(}\DecValTok{100}\NormalTok{, }\DecValTok{100}\NormalTok{, }\DataTypeTok{autocorr_range =} \DecValTok{1}\NormalTok{, }\DataTypeTok{user_seed =} \DecValTok{1000}\NormalTok{)}
\end{Highlighting}
\end{Shaded}

\begin{verbatim}
## NOTE: simulation is performed with fixed random seed 1000.
## Set 'RFoptions(seed=NA)' to make the seed arbitrary.
\end{verbatim}

\begin{Shaded}
\begin{Highlighting}[]
\NormalTok{grf_010 <-}\StringTok{ }\KeywordTok{nlm_gaussianfield}\NormalTok{(}\DecValTok{100}\NormalTok{, }\DecValTok{100}\NormalTok{, }\DataTypeTok{autocorr_range =} \DecValTok{10}\NormalTok{, }\DataTypeTok{user_seed =} \DecValTok{1000}\NormalTok{)}
\end{Highlighting}
\end{Shaded}

\begin{verbatim}
## NOTE: simulation is performed with fixed random seed 1000.
## Set 'RFoptions(seed=NA)' to make the seed arbitrary.
\end{verbatim}

\begin{Shaded}
\begin{Highlighting}[]
\NormalTok{grf_100 <-}\StringTok{ }\KeywordTok{nlm_gaussianfield}\NormalTok{(}\DecValTok{100}\NormalTok{, }\DecValTok{100}\NormalTok{, }\DataTypeTok{autocorr_range =} \DecValTok{100}\NormalTok{, }\DataTypeTok{user_seed =} \DecValTok{1000}\NormalTok{)}
\end{Highlighting}
\end{Shaded}

\begin{verbatim}
## NOTE: simulation is performed with fixed random seed 1000.
## Set 'RFoptions(seed=NA)' to make the seed arbitrary.
\end{verbatim}

\begin{Shaded}
\begin{Highlighting}[]
\NormalTok{env <-}\StringTok{ }\KeywordTok{brick}\NormalTok{(null, uniform, random, distance_gradient_center, edge_gradient, oscillation, grf_001, grf_010, grf_100)}
\end{Highlighting}
\end{Shaded}

\includegraphics{evolution-of-routes-and-paths_files/figure-latex/unnamed-chunk-9-1.pdf}

\subsection{Spatial Interaction Model}\label{spatial-interaction-model}

\begin{Shaded}
\begin{Highlighting}[]
\NormalTok{interact <-}\StringTok{ }\NormalTok{function(net)\{}
  \NormalTok{net %E>%}
\StringTok{    }\KeywordTok{mutate}\NormalTok{(}\DataTypeTok{interaction_strength =} \KeywordTok{.N}\NormalTok{()$attractiveness[to] ^}\StringTok{ }\NormalTok{alpha *}\StringTok{ }\KeywordTok{exp}\NormalTok{(-beta *}\StringTok{ }\NormalTok{distance))  %N>%}
\StringTok{    }\KeywordTok{mutate}\NormalTok{(}\DataTypeTok{outflow =} \NormalTok{population /}\StringTok{ }\KeywordTok{centrality_degree}\NormalTok{(}\DataTypeTok{weights =} \NormalTok{interaction_strength, }\DataTypeTok{mode =} \StringTok{'out'}\NormalTok{, }\DataTypeTok{loops =} \OtherTok{FALSE}\NormalTok{)) %E>%}
\StringTok{    }\KeywordTok{mutate}\NormalTok{(}\DataTypeTok{flow =} \KeywordTok{.N}\NormalTok{()$outflow[from] *}\StringTok{ }\NormalTok{interaction_strength) %N>%}
\StringTok{    }\KeywordTok{mutate}\NormalTok{(}\DataTypeTok{inflow =} \KeywordTok{centrality_degree}\NormalTok{(}\DataTypeTok{weights =} \NormalTok{flow, }\DataTypeTok{mode =} \StringTok{'in'}\NormalTok{, }\DataTypeTok{loops =} \NormalTok{F),}
           \DataTypeTok{attractiveness =} \NormalTok{attractiveness +}\StringTok{ }\NormalTok{.}\DecValTok{01} \NormalTok{*}\StringTok{ }\NormalTok{(inflow  -}\StringTok{ }\NormalTok{attractiveness),}
           \DataTypeTok{population =} \NormalTok{n *}\StringTok{ }\NormalTok{pop_start *}\StringTok{ }\NormalTok{attractiveness /}\StringTok{ }\KeywordTok{sum}\NormalTok{(attractiveness))}
\NormalTok{\}}
\end{Highlighting}
\end{Shaded}

\begin{Shaded}
\begin{Highlighting}[]
\NormalTok{nystuen_dacey <-}\StringTok{ }\NormalTok{function(net)\{}
  \NormalTok{net %E>%}
\StringTok{  }\KeywordTok{group_by}\NormalTok{(from) %>%}
\StringTok{  }\KeywordTok{filter}\NormalTok{(flow ==}\StringTok{ }\KeywordTok{max}\NormalTok{(flow), }\KeywordTok{.N}\NormalTok{()$population[from] <}\StringTok{ }\KeywordTok{.N}\NormalTok{()$population[to]) %N>%}
\StringTok{  }\KeywordTok{mutate}\NormalTok{(}\DataTypeTok{terminal =} \KeywordTok{node_is_sink}\NormalTok{()) %>%}
\StringTok{  }\NormalTok{ungroup}
\NormalTok{\}}
\end{Highlighting}
\end{Shaded}

\section{Results}\label{results}

\begin{Shaded}
\begin{Highlighting}[]
\NormalTok{sim_length <-}\StringTok{ }\DecValTok{100} \CommentTok{# number of time steps}

\NormalTok{sim <-}\StringTok{ }\KeywordTok{accumulate}\NormalTok{(}\DecValTok{1}\NormalTok{:sim_length, ~}\KeywordTok{interact}\NormalTok{(.x), }\DataTypeTok{.init =} \NormalTok{net) %>%}
\StringTok{  }\NormalTok{.[}\DecValTok{2}\NormalTok{:sim_length] %>%}\StringTok{ }\CommentTok{# remove the initial state}
\StringTok{  }\KeywordTok{map}\NormalTok{(nystuen_dacey) %>%}
\StringTok{  }\KeywordTok{map}\NormalTok{(as_tibble) %>%}
\StringTok{  }\KeywordTok{bind_rows}\NormalTok{(}\DataTypeTok{.id =} \StringTok{'time'}\NormalTok{) %>%}
\StringTok{  }\KeywordTok{mutate}\NormalTok{(}\DataTypeTok{time =} \KeywordTok{as.numeric}\NormalTok{(time))}
\end{Highlighting}
\end{Shaded}

\includegraphics{evolution-of-routes-and-paths_files/figure-latex/unnamed-chunk-12-1.pdf}
\includegraphics{evolution-of-routes-and-paths_files/figure-latex/unnamed-chunk-12-2.pdf}
\includegraphics{evolution-of-routes-and-paths_files/figure-latex/unnamed-chunk-12-3.pdf}
\includegraphics{evolution-of-routes-and-paths_files/figure-latex/unnamed-chunk-12-4.pdf}
\includegraphics{evolution-of-routes-and-paths_files/figure-latex/unnamed-chunk-12-5.pdf}
\includegraphics{evolution-of-routes-and-paths_files/figure-latex/unnamed-chunk-12-6.pdf}
\includegraphics{evolution-of-routes-and-paths_files/figure-latex/unnamed-chunk-12-7.pdf}
\includegraphics{evolution-of-routes-and-paths_files/figure-latex/unnamed-chunk-12-8.pdf}
\includegraphics{evolution-of-routes-and-paths_files/figure-latex/unnamed-chunk-12-9.pdf}
\includegraphics{evolution-of-routes-and-paths_files/figure-latex/unnamed-chunk-12-10.pdf}
\includegraphics{evolution-of-routes-and-paths_files/figure-latex/unnamed-chunk-12-11.pdf}
\includegraphics{evolution-of-routes-and-paths_files/figure-latex/unnamed-chunk-12-12.pdf}
\includegraphics{evolution-of-routes-and-paths_files/figure-latex/unnamed-chunk-12-13.pdf}
\includegraphics{evolution-of-routes-and-paths_files/figure-latex/unnamed-chunk-12-14.pdf}
\includegraphics{evolution-of-routes-and-paths_files/figure-latex/unnamed-chunk-12-15.pdf}
\includegraphics{evolution-of-routes-and-paths_files/figure-latex/unnamed-chunk-12-16.pdf}
\includegraphics{evolution-of-routes-and-paths_files/figure-latex/unnamed-chunk-12-17.pdf}
\includegraphics{evolution-of-routes-and-paths_files/figure-latex/unnamed-chunk-12-18.pdf}
\includegraphics{evolution-of-routes-and-paths_files/figure-latex/unnamed-chunk-12-19.pdf}
\includegraphics{evolution-of-routes-and-paths_files/figure-latex/unnamed-chunk-12-20.pdf}
\includegraphics{evolution-of-routes-and-paths_files/figure-latex/unnamed-chunk-12-21.pdf}
\includegraphics{evolution-of-routes-and-paths_files/figure-latex/unnamed-chunk-12-22.pdf}
\includegraphics{evolution-of-routes-and-paths_files/figure-latex/unnamed-chunk-12-23.pdf}
\includegraphics{evolution-of-routes-and-paths_files/figure-latex/unnamed-chunk-12-24.pdf}
\includegraphics{evolution-of-routes-and-paths_files/figure-latex/unnamed-chunk-12-25.pdf}
\includegraphics{evolution-of-routes-and-paths_files/figure-latex/unnamed-chunk-12-26.pdf}
\includegraphics{evolution-of-routes-and-paths_files/figure-latex/unnamed-chunk-12-27.pdf}
\includegraphics{evolution-of-routes-and-paths_files/figure-latex/unnamed-chunk-12-28.pdf}
\includegraphics{evolution-of-routes-and-paths_files/figure-latex/unnamed-chunk-12-29.pdf}
\includegraphics{evolution-of-routes-and-paths_files/figure-latex/unnamed-chunk-12-30.pdf}
\includegraphics{evolution-of-routes-and-paths_files/figure-latex/unnamed-chunk-12-31.pdf}
\includegraphics{evolution-of-routes-and-paths_files/figure-latex/unnamed-chunk-12-32.pdf}
\includegraphics{evolution-of-routes-and-paths_files/figure-latex/unnamed-chunk-12-33.pdf}
\includegraphics{evolution-of-routes-and-paths_files/figure-latex/unnamed-chunk-12-34.pdf}
\includegraphics{evolution-of-routes-and-paths_files/figure-latex/unnamed-chunk-12-35.pdf}
\includegraphics{evolution-of-routes-and-paths_files/figure-latex/unnamed-chunk-12-36.pdf}
\includegraphics{evolution-of-routes-and-paths_files/figure-latex/unnamed-chunk-12-37.pdf}
\includegraphics{evolution-of-routes-and-paths_files/figure-latex/unnamed-chunk-12-38.pdf}
\includegraphics{evolution-of-routes-and-paths_files/figure-latex/unnamed-chunk-12-39.pdf}
\includegraphics{evolution-of-routes-and-paths_files/figure-latex/unnamed-chunk-12-40.pdf}
\includegraphics{evolution-of-routes-and-paths_files/figure-latex/unnamed-chunk-12-41.pdf}
\includegraphics{evolution-of-routes-and-paths_files/figure-latex/unnamed-chunk-12-42.pdf}
\includegraphics{evolution-of-routes-and-paths_files/figure-latex/unnamed-chunk-12-43.pdf}
\includegraphics{evolution-of-routes-and-paths_files/figure-latex/unnamed-chunk-12-44.pdf}
\includegraphics{evolution-of-routes-and-paths_files/figure-latex/unnamed-chunk-12-45.pdf}
\includegraphics{evolution-of-routes-and-paths_files/figure-latex/unnamed-chunk-12-46.pdf}
\includegraphics{evolution-of-routes-and-paths_files/figure-latex/unnamed-chunk-12-47.pdf}
\includegraphics{evolution-of-routes-and-paths_files/figure-latex/unnamed-chunk-12-48.pdf}
\includegraphics{evolution-of-routes-and-paths_files/figure-latex/unnamed-chunk-12-49.pdf}
\includegraphics{evolution-of-routes-and-paths_files/figure-latex/unnamed-chunk-12-50.pdf}
\includegraphics{evolution-of-routes-and-paths_files/figure-latex/unnamed-chunk-12-51.pdf}
\includegraphics{evolution-of-routes-and-paths_files/figure-latex/unnamed-chunk-12-52.pdf}
\includegraphics{evolution-of-routes-and-paths_files/figure-latex/unnamed-chunk-12-53.pdf}
\includegraphics{evolution-of-routes-and-paths_files/figure-latex/unnamed-chunk-12-54.pdf}
\includegraphics{evolution-of-routes-and-paths_files/figure-latex/unnamed-chunk-12-55.pdf}
\includegraphics{evolution-of-routes-and-paths_files/figure-latex/unnamed-chunk-12-56.pdf}
\includegraphics{evolution-of-routes-and-paths_files/figure-latex/unnamed-chunk-12-57.pdf}
\includegraphics{evolution-of-routes-and-paths_files/figure-latex/unnamed-chunk-12-58.pdf}
\includegraphics{evolution-of-routes-and-paths_files/figure-latex/unnamed-chunk-12-59.pdf}
\includegraphics{evolution-of-routes-and-paths_files/figure-latex/unnamed-chunk-12-60.pdf}
\includegraphics{evolution-of-routes-and-paths_files/figure-latex/unnamed-chunk-12-61.pdf}
\includegraphics{evolution-of-routes-and-paths_files/figure-latex/unnamed-chunk-12-62.pdf}
\includegraphics{evolution-of-routes-and-paths_files/figure-latex/unnamed-chunk-12-63.pdf}
\includegraphics{evolution-of-routes-and-paths_files/figure-latex/unnamed-chunk-12-64.pdf}
\includegraphics{evolution-of-routes-and-paths_files/figure-latex/unnamed-chunk-12-65.pdf}
\includegraphics{evolution-of-routes-and-paths_files/figure-latex/unnamed-chunk-12-66.pdf}
\includegraphics{evolution-of-routes-and-paths_files/figure-latex/unnamed-chunk-12-67.pdf}
\includegraphics{evolution-of-routes-and-paths_files/figure-latex/unnamed-chunk-12-68.pdf}
\includegraphics{evolution-of-routes-and-paths_files/figure-latex/unnamed-chunk-12-69.pdf}
\includegraphics{evolution-of-routes-and-paths_files/figure-latex/unnamed-chunk-12-70.pdf}
\includegraphics{evolution-of-routes-and-paths_files/figure-latex/unnamed-chunk-12-71.pdf}
\includegraphics{evolution-of-routes-and-paths_files/figure-latex/unnamed-chunk-12-72.pdf}
\includegraphics{evolution-of-routes-and-paths_files/figure-latex/unnamed-chunk-12-73.pdf}
\includegraphics{evolution-of-routes-and-paths_files/figure-latex/unnamed-chunk-12-74.pdf}
\includegraphics{evolution-of-routes-and-paths_files/figure-latex/unnamed-chunk-12-75.pdf}
\includegraphics{evolution-of-routes-and-paths_files/figure-latex/unnamed-chunk-12-76.pdf}
\includegraphics{evolution-of-routes-and-paths_files/figure-latex/unnamed-chunk-12-77.pdf}
\includegraphics{evolution-of-routes-and-paths_files/figure-latex/unnamed-chunk-12-78.pdf}
\includegraphics{evolution-of-routes-and-paths_files/figure-latex/unnamed-chunk-12-79.pdf}
\includegraphics{evolution-of-routes-and-paths_files/figure-latex/unnamed-chunk-12-80.pdf}
\includegraphics{evolution-of-routes-and-paths_files/figure-latex/unnamed-chunk-12-81.pdf}
\includegraphics{evolution-of-routes-and-paths_files/figure-latex/unnamed-chunk-12-82.pdf}
\includegraphics{evolution-of-routes-and-paths_files/figure-latex/unnamed-chunk-12-83.pdf}
\includegraphics{evolution-of-routes-and-paths_files/figure-latex/unnamed-chunk-12-84.pdf}
\includegraphics{evolution-of-routes-and-paths_files/figure-latex/unnamed-chunk-12-85.pdf}
\includegraphics{evolution-of-routes-and-paths_files/figure-latex/unnamed-chunk-12-86.pdf}
\includegraphics{evolution-of-routes-and-paths_files/figure-latex/unnamed-chunk-12-87.pdf}
\includegraphics{evolution-of-routes-and-paths_files/figure-latex/unnamed-chunk-12-88.pdf}
\includegraphics{evolution-of-routes-and-paths_files/figure-latex/unnamed-chunk-12-89.pdf}
\includegraphics{evolution-of-routes-and-paths_files/figure-latex/unnamed-chunk-12-90.pdf}
\includegraphics{evolution-of-routes-and-paths_files/figure-latex/unnamed-chunk-12-91.pdf}
\includegraphics{evolution-of-routes-and-paths_files/figure-latex/unnamed-chunk-12-92.pdf}
\includegraphics{evolution-of-routes-and-paths_files/figure-latex/unnamed-chunk-12-93.pdf}
\includegraphics{evolution-of-routes-and-paths_files/figure-latex/unnamed-chunk-12-94.pdf}
\includegraphics{evolution-of-routes-and-paths_files/figure-latex/unnamed-chunk-12-95.pdf}
\includegraphics{evolution-of-routes-and-paths_files/figure-latex/unnamed-chunk-12-96.pdf}
\includegraphics{evolution-of-routes-and-paths_files/figure-latex/unnamed-chunk-12-97.pdf}
\includegraphics{evolution-of-routes-and-paths_files/figure-latex/unnamed-chunk-12-98.pdf}
\includegraphics{evolution-of-routes-and-paths_files/figure-latex/unnamed-chunk-12-99.pdf}
\includegraphics{evolution-of-routes-and-paths_files/figure-latex/unnamed-chunk-12-100.pdf}

Here are two sample references: Feynman and Vernon Jr. (1963; Dirac
1953).

\section{Discussion}\label{discussion}

\section{Conclusion}\label{conclusion}

\section*{References}\label{references}
\addcontentsline{toc}{section}{References}

\hypertarget{refs}{}
\hypertarget{ref-Dirac1953888}{}
Dirac, P.A.M. 1953. ``The Lorentz Transformation and Absolute Time.''
\emph{Physica} 19 (1---12): 888--96.
doi:\href{https://doi.org/10.1016/S0031-8914(53)80099-6}{10.1016/S0031-8914(53)80099-6}.

\hypertarget{ref-Feynman1963118}{}
Feynman, R.P, and F.L Vernon Jr. 1963. ``The Theory of a General Quantum
System Interacting with a Linear Dissipative System.'' \emph{Annals of
Physics} 24: 118--73.
doi:\href{https://doi.org/10.1016/0003-4916(63)90068-X}{10.1016/0003-4916(63)90068-X}.

\end{document}


